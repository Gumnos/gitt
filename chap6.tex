% chap6.tex - Week 6
\cleardoublepage
%\phantomsection
\chapter{Week 6}

\section{Day 1 - ``My private little stash''}
\subsection{Getting interrrupted}

We're getting close to the point where we can really start using Git as we originally intended, the team at Tamagoyaki Inc. are also getting much more aquainted with the operations of both version control and Git in particular.  Unfortunately things don't always go as smoothly as we would like.  The following scenario demonstrates just this.

\begin{trenches}
``Yeh but Martha, I need you to work on this fix now!  Not in five minutes.''

``Klaus I'm kinda in the middle of something else yet, and I'm not ready to commit,'' said Martha, feeling a little concerned.  She was used to Klaus making demands on his time, but she had spent a while working on this particular fix for John and she just wasn't ready to finish up yet.

``You could always use the stash feature.''  It was Rob again, it seemed as if this young user had cottoned on to Git quicker than most of the seasoned developers.''

Klaus seemed unimpressed, ``What the heck does that do?''

``Allows you to move your changes to somewhere else until you are ready to finish them.''

``Please Rob,'' started Martha, ``Can you show me what you mean?  Sounds exactly like what I need''
\end{trenches}

Sometimes you could be in the middle of something when another really important task comes up.  When this happens you are often required to drop everything and carry on with another task.  Often this can be quite difficult.  You may be in a development branch, but unable or unwilling to make a commit at this stage.  So how can we deal with this?

Well, one way of dealing with this situation, and this is not necessarily the best way, is to do the steps outlined below;

\begin{enumerate}
\item Make a new branch called something like WIP
\item Pull changes into this new branch
\item Commit changes into WIP branch
\item Switch back to the branch we need to work on
\item Make our changes and commit them
\item Merge in our WIP branch on top with the \texttt{--no-commit} option
\item Delete the WIP branch
\item Continue development in our original branch
\end{enumerate}

To me, that seems like a lot of work.  OK, the benefit is a fairly awesome one, but at the cost of considerable command line hackery to get there.  It would be nice if there was an easy way to do the above, and though someone of you maybe screaming something like \emph{shell script} right about now, you can rest your fingers.  We actualy have another tool in the Git toolbox to help us out.

The \texttt{git stash} command is used for exactly these situations.  Let us make a quick example of how to use it in our test repository.  To begin with we are going to make a few changes to the \textbf{master} branch, before we are interrupted.

\begin{Verbatim}[frame=leftline,framerule=1mm,fontsize=\relsize{-3}] 
john@akira:~/coderepo$ git checkout master
Already on 'master'
john@akira:~/coderepo$ echo "Number strings rule 1234" >> another_file 
john@akira:~/coderepo$ 
\end{Verbatim}

So now we are interrupted and we have to make some important, and urgent changes to our \textbf{master} branch.  The example that follows is one way that we could use the \texttt{git stash} command to help us out.  

\begin{Verbatim}[frame=leftline,framerule=1mm,fontsize=\relsize{-3}] 
john@akira:~/coderepo$ git stash
Saved working directory and index state WIP on master: f91f31e Added another file
HEAD is now at f91f31e Added another file
john@akira:~/coderepo$ nano my_third_committed_file 
john@akira:~/coderepo$ echo "Some mega important changes" >> my_third_committed_file 
john@akira:~/coderepo$ git commit -a -m 'Important Update'
[master 6ab599d] Important Update
 1 files changed, 1 insertions(+), 0 deletions(-)
john@akira:~/coderepo$ 
\end{Verbatim}

So using the \texttt{git stash} command, we have squirreled all of our developmental changes away into a \emph{stash}.  Now we have completed the mega important change that just could not wait, we are ready to pull our changes back from the stash.  First let us see just what the stash contains.

\begin{Verbatim}[frame=leftline,framerule=1mm,fontsize=\relsize{-3}] 
john@akira:~/coderepo$ git stash list
stash@{0}: WIP on master: f91f31e Added another file
john@akira:~/coderepo$ git stash show stash@{0}
 another_file |    1 +
 1 files changed, 1 insertions(+), 0 deletions(-)
john@akira:~/coderepo$ git stash show stash@{0} -p
diff --git a/another_file b/another_file
index dba885d..b3a5cc5 100644
--- a/another_file
+++ b/another_file
@@ -1 +1,2 @@
 New stuff
+Number strings rule 1234
john@akira:~/coderepo$ 
\end{Verbatim}

As you can see, the \texttt{-p} option to the \texttt{git stash show} command shows us exactly what is contained in the stash.  We can apply this stash by running the following;

\begin{Verbatim}[frame=leftline,framerule=1mm,fontsize=\relsize{-3}] 
john@akira:~/coderepo$ git stash apply stash@{0} 
# On branch master
# Changed but not updated:
#   (use "git add <file>..." to update what will be committed)
#   (use "git checkout -- <file>..." to discard changes in working directory)
#
#	modified:   another_file
#
# Untracked files:
#   (use "git add <file>..." to include in what will be committed)
#
#	temp_file
no changes added to commit (use "git add" and/or "git commit -a")
john@akira:~/coderepo$ git commit -a -m 'Continued Development'
[master 725c961] Continued Development
 1 files changed, 1 insertions(+), 0 deletions(-)
john@akira:~/coderepo$ git stash list
stash@{0}: WIP on master: f91f31e Added another file
john@akira:~/coderepo$ 
\end{Verbatim}

Our stash has been applied to the current branch, in our case \textbf{master}, and we have gone ahead and committed these changes into the repository.  Interestingly, our stash still exists.  If we had used the \texttt{git stash pop} command instead of the \texttt{git stash apply}, our stash would have been removed.  Of course we can have multiple stashes in our repository.  For completeness sake, let us learn how to manually remove a stash, and take a look at how our repository looks with our semi-graphical \texttt{git log}

\begin{Verbatim}[frame=leftline,framerule=1mm,fontsize=\relsize{-3}] 
john@akira:~/coderepo$ git stash drop stash@{0} 
Dropped stash@{0} (e994874521ebea57b798150d3936e41e50c49386)
john@akira:~/coderepo$ git log --graph --pretty=oneline --all --abbrev-commit --decorate
* 725c961 (HEAD, master) Continued Development
* 6ab599d Important Update
| * 4fdfe36 (zaney) Made another awesome change
| * 46c2390 Made an awesome change
|/  
| * f0cbf32 (wonderful) Fantastic new feature
|/  
* f91f31e Added another file
* 8d5140a (v2.0) Added two new files
* b5c84a5 Removed a few files
* fa65f06 (v1.0a) Messed with a few files
* 6ca160c Finished adding initial files
* e86ddea (v0.9) Made a few changes to first and second files
* 8820692 My First Ever Commit
john@akira:~/coderepo$ 
\end{Verbatim}

Using the \texttt{git stash drop} command, we can drop a single stash from the list.


\section{Day 2 - ``What?! No backup?''}
\subsection{Attack of the clones}

We now know about basic branching and merging.  At this stage, there is on important topic we must cover briefly, and this is subject of cloning.  Cloning allows you to make a complete copy of your repository, including all of its history and all of the branches.  Let us take a look at a situation which could make use of cloning.

\begin{trenches}
``So John,'' started Rob, ``Just how are we backing up the repository at the moment?''

John thought for a moment before replying.  He knew what Rob was getting at, but he hadn't expected Rob to bring the question up in front of Markus.  In truth he had forgotten all about it.  He turned to Markus.

``I'll be honest Markus.  Currently the repository isn't being backed up, but then we are running in parallel with the old system, so it would be the end of the world if we lost it.''

Markus nodded and smiled.  It seemed that John had gotten away with it for now and with that, Klaus shot Rob a piercing glance.  

``Team,'' began Markus, ``we need a definitive way of backing up the new Git repository, and I'd like it done before the end of the day.''  He pointed at a document on the table.  ``This project document has been approved by Wayne, and in there it states we will have a defined backup strategy.  Please don't let me down.''

\begin{center} * * * \end{center}

``Rob you really got to be careful about things like that,'' Klaus said to one of the younger members of the team.  ``You really showed John up in there.''  

``Yeh,'' said Rob, ``I realise that now.''  He stood with his back against the wall and tapped his fingers against the painted surface.  ``I was wondering, do you think John would let me look at the backup system for him as a way of apologising.''

Klaus smiled, it seemed Rob was finally understanding things, ``Go ask him,'' said Klaus, ``He's so snowed under with the BurnForce release that he'll probably let you implement it too''

``Right'' nodded Rob and off he went.
\end{trenches}

So cloning is an excellent way of taking a copy of our repository, in essence it is a simple way of taking a backup of our repository and keeping it up to date.  Obviously we could just take the files and copy them, but a better way of doing this is by utilising the \texttt{git clone} command.  When cloning a repository, we take then entire structure and replicate it, creating an exact copy of the data in an alternative location.  Well, that's what cloning means isn't it?

The git clone tool doesn't just copy the data though, it does several other things.  Let us create a clone of our test repository to another local location.  In this case, we are going to clone the repository into a folder called \texttt{coderepo-clone}.

\begin{Verbatim}[frame=leftline,framerule=1mm,fontsize=\relsize{-3}] 
john@akira:~$ git clone coderepo coderepo-clone
Initialized empty Git repository in /home/john/coderepo-clone/.git/
john@akira:~$ 
\end{Verbatim}

\begin{Verbatim}[frame=leftline,framerule=1mm,fontsize=\relsize{-3}] 
john@akira:~$ cd coderepo-clone
john@akira:~/coderepo-clone$ ls
another_file  my_third_committed_file  newfile1  newfile2
john@akira:~/coderepo-clone$
\end{Verbatim}

\begin{Verbatim}[frame=leftline,framerule=1mm,fontsize=\relsize{-3}] 
john@akira:~/coderepo-clone$ git status
# On branch master
nothing to commit (working directory clean)
john@akira:~/coderepo-clone$ 
\end{Verbatim}

Let us take a few moments to see what has happened here.  It would appear that our data has been copied successfully.  If you were to run a \texttt{git log} on this dir, you would see all the previous log messages for all our previous commits.  You can also see that if we run \texttt{git status} that we are on the same branch here that we were in our repository when we left it.  In this case, we are in the \textbf{master} branch.

\begin{trenches}
``Klaus, this doesn't make sense though.  I cloned my repo, but my branches have vanished,'' started Rob, rather worriedly.

``I'm sure they haven't gone anywhere,'' shouted John.  

Rob's reply was quick and certain, ``They have!  Run a git branch on your clone and see for yourself''

``Oh!''

Rob smiled the smug smile of a man who is pleased he was right.

\begin{center} * * * \end{center}

``Rob?'' asked John tentatively, aware of his previous state of anguish.

His reply indicated a less than stressed demeanour, ``Sup, dude?''.

John was pleasantly suprised, ``Did you sort out the branch issue?''

Rob begain walking over to John, ``Yeh, try running git branch with the r parameter.''

\end{trenches}

You may be thinking that we now have a detached copy of everything in our repository.  That's not exactly accurate.  Let us run the command that Rob suggested and see what is happening.  

\begin{Verbatim}[frame=leftline,framerule=1mm,fontsize=\relsize{-3}] 
john@akira:~/coderepo-clone$ git branch
* master
john@akira:~/coderepo-clone$ git branch -r
  origin/HEAD -> origin/master
  origin/master
  origin/wacky
john@akira:~/coderepo-clone$ 
\end{Verbatim}

Whilst we have all the objects in our repository, we do not yet have our branches set up locally.  They are available, but they have not been setup.  To explain this we need to introduce the concept of remote tracking branches.  We are familiar with local branches.  Local branches allowed us to make commits locally in a safe and separated environment.  Remote tracking branches are links to remote branches which allow us to track the development of that branch and bring it in to a local one if we so desire.

With a remote tracking branch, we can pull in the changes from the remote repository

Now we have a complete copy of our repository in another location.  At the moment we have created this clone on the same machine that our original is.  This isn't really a very good idea for backup purposes.  Git supplies several means with which to talk to a remote machine, but by far the most common of these is to utilise the SSH protocol.  SSH is a secure, encrypted way to communicate with a remote repository.  

If we assume that for a moment that our user john has now moved to another machine

In the next step, we are assuming that we have another machine where our user john has SSH access, and we are going to clone our repository there.

%POSSIBLY Talk about the SHARED and REFERENCE in a breakout
%POSSIBLY TALK ABOUT SSH
%ssh://[user@]host.xz[:port]/path/to/repo.git/

\clearpage

\section{Summary - John's Notes}
\subsection{Commands}
\begin{itemize}

\item\texttt{git stash} - Short for \texttt{git stash save}, creates a stash of local modifications

\item\texttt{git stash apply <stash\_name>} - Applies a stash back onto a branch

\item\texttt{git stash drop <stash\_name>} - Remove a stash from the stash list

\item\texttt{git stash list} - Show a list of current stashes

\item\texttt{git stash show <stash\_name>} - Show information about a specific stash

\item\texttt{git stash show <stash\_name> -p} - Show the contents of a stash

\end{itemize}

\subsection{Terminology}
\begin{itemize}
\item\textbf{Stash} - A temporary storage of local modifications that can be brough back onto the branch at a later date
\end{itemize}
