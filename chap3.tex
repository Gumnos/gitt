% chap3.tex - Week 3
\cleardoublepage
%\phantomsection
\chapter{Week 3}
\section{Day 1 - ``But how do I see what's going on?''}
\subsection{Logging in Git}

Perhaps the best feature of a version control system is the level of accountability that it offers if set up correctly.  What do we mean by this?  People often mistake the word \textbf{accountability} for the word \textbf{blame}.  This is not true at all.  Accountability is key in understanding the events that led up to a particular bug being introduced, or a situation occuring.  How this is dealt with, is a up to the management teams, but accountability should not be something that is revered, it should be something that is looked upon as a tool to help define the cause of a problem.

By its very nature, a version control system is also a logging system.  Every time we committed something into the repository in the last chapter, we supplied a log message.  In fact, if we don't supply a commit message, let us see what happens.

\begin{Verbatim}[frame=single,fontsize=\relsize{-3}] 
john@akira:~/coderepo$ git status
# On branch master
# Untracked files:
#   (use "git add <file>..." to include in what will be 
committed)
#
#	my_third_committed_file
nothing added to commit but untracked files present (use "git 
add" to track)
john@akira:~/coderepo$ git add my_third_committed_file
john@akira:~/coderepo$ git commit -a -m ''
Aborting commit due to empty commit message.
john@akira:~/coderepo$ 
\end{Verbatim}

So, Git will actually not allow you to commit with a blank message.  This is actually fantastic news, as people are far less likely to write a useless message than they are a blank one.  It is very important that when using a version control system you write in a useful commit message.  If you fixed a bug, say so.  If you added a new function, why not put that in too.  When someone wants to find out what a certain commit was for, or even when you come back to the project six months later and realise you've forgotten everything, log messages are crucial in piecing back together a history of development.

\begin{trenches}
``So John, I've been committing and all that,'' started Rob, ``but how do I see the history of what I have done.''

``It's really pretty simple,'' replied John, ``But it really depends on what you want to know.''
Rob placed his thumb and forefinger onto his chin.  ``Well, for now, I just want to see a list of all of my commits.''

``That one's the simplest of all.''
\end{trenches}

At its simplest, \texttt{git log} will give an output of all of the commits that have been applied to the current branch.  Depending on what type of machine you are using it on, the output from \texttt{git log} will be navigatable, usually using the up and down arrows, with 'q' used to quit.  Let's have a quick look at the output of our test repository and see what the log messages look like.

\begin{Verbatim}[frame=single,fontsize=\relsize{-3}]
john@akira:~/coderepo$ git log
commit 6ca160c7226731bf80973fc5bc81f6b9beda7795
Author: John Haskins <john.haskins@tamagoyakiinc.koala>
Date:   Mon Feb 21 20:59:32 2011 +0000

    Finished adding initial files

commit e86ddea25341a75275d316d8ca545aa7c73e97b3
Author: John Haskins <john.haskins@tamagoyakiinc.koala>
Date:   Mon Feb 21 20:06:57 2011 +0000

    Made a few changes to first and second files

commit 88206926cb60aed53d21ede69f9ca5b7c69cb983
Author: John Haskins <john.haskins@tamagoyakiinc.koala>
Date:   Sat Feb 19 09:23:47 2011 +0000

    My First Ever Commit
john@akira:~/coderepo$ 
\end{Verbatim}

The \texttt{git log} command shows us a chronological list of all of the commits to the repository and also gives us several more important pieces of information.  In total there are four pieces of information displayed by default.

\begin{itemize}
\item \textbf{commit} - This is the SHA-1 hash of the commit object that is stored inside the repository.  You can find more information about this in the \emph{What's inside the Git repository?} section \emph{Week 2}.  This is how we refer to the commit.  If someone asked you in what commit you \emph{Made a few changes to first and second files}, you could reply that you did that in commit e86dd.  As explained earlier, it is good to remember that you don't need to remember or type out the whole \textbf{e86ddea25341a75275d316d8ca545aa7c73e97b3}, only the first part is required.  Generally, the first five characters will do.
\item \textbf{Author} - This is the name and email address of the author of the commit.  When we begin to look at merging, you will see that the author of a commit, is not necessarily the \emph{committer} of the commit.  If you want to find out more about how to set these options, see the breakout box in this Week, called \emph{Changing your identity}.
\item \textbf{Date} - The date is simply the date at which the commit was created.  Again, note that when we start looking at merging, the date will be the date the commit was created, not the date it was merged into the repository.
\item \textbf{Commit Message} - This is the log message that was added along with the commit when it was created.  Hopefully you can now see how important it is to create useful and meaningful messages in here.
\end{itemize}

       %git log --no-merges
           %Show the whole commit history, but skip any merges

       %git log v2.6.12.. include/scsi drivers/scsi
           %Show all commits since version v2.6.12 that changed any file in the include/scsi or drivers/scsi subdirectories

       %git log --since="2 weeks ago" -- gitk
           %Show the changes during the last two weeks to the file gitk. The "--" is necessary to avoid confusion with the branch named gitk

       %git log --name-status release..test
           %Show the commits that are in the "test" branch but not yet in the "release" branch, along with the list of paths each commit modifies.

       %git log --follow builtin-rev-list.c
           %Shows the commits that changed builtin-rev-list.c, including those commits that occurred before the file was given its present name.

       %git log --branches --not --remotes=origin
           %Shows all commits that are in any of local branches but not in any of remote tracking branches for origin (what you have that origin
           %doesn’t).

       %git log master --not --remotes=*/master
           %Shows all commits that are in local master but not in any remote repository master branches.

       %git log -p -m --first-parent
           %Shows the history including change diffs, but only from the "main branch" perspective, skipping commits that come from merged branches, and
           %showing full diffs of changes introduced by the merges. This makes sense only when following a strict policy of merging all topic branches
           %when staying on a single integration branch.

\section{Day 2 - ``But I need more information''}
\subsection{Digging a little deeper}

\begin{trenches}
``I know John, and next time I will make a note of it, but right now, I'd really like to know where this file got changed,'' Klaus pointed at the piece of paper containing a print out, ``specifically when this function was introduced.''

John smiled.  His hands danced over the keyboard as he finished compiling an email.  ``And you've no idea when this was added at all?'' he asked.

``No, sorry John, I don't.''  He pondered, ``I guess I could write a script to untar all the versions we've created in the last week and search through them.''  He sighed, ``Can't the wonderful Git help us out here?''

A head popped up over the cubicle wall.  ``You wanna find out when a function was introduced to a file?''  It was Rob.  ``After John showed me the basics, I went and read up on it a little.  Git has some really powerful searching within the log tool.''

``Well come on then,'' blurted Klaus, ``Don't keep me hanging on.''

A chime of the popular 1966 hit sprang out in the office.

Klaus pulled a hand down over his face, ``Oh don't you all start!''
\end{trenches}

Git can actually do some rather powerful searching to assist a developer in their daily tasks.  It would have been useful if the particular item that was being searched for had been included in the log, but sometimes, things either get missed, or there are just too many changes introduced in one commit to list them all.

In these instances, the \texttt{git log -S<string>} command comes to our aid.  This command will search through the commits in a repository and will return a list of commits which introduced or removed a specific string into the repository.  First of all, let's run this against our test repository.

\begin{Verbatim}[frame=single,fontsize=\relsize{-3}] 
john@akira:~/coderepo$ git log -SChange1
commit e86ddea25341a75275d316d8ca545aa7c73e97b3
Author: John Haskins <john.haskins@tamagoyakiinc.koala>
Date:   Mon Feb 21 20:06:57 2011 +0000

    Made a few changes to first and second files
john@akira:~/coderepo$ 
\end{Verbatim}

As you can see, \texttt{git log} has shown us the commit that instantiated the change.  As you can imagine, when using a large code base, this tool can be invaluable.  It allows us to pinpoint a specific moment when a certain string of text entered the repository.  When running this against a very large repository, this could take a long time, and so the ability to shrink the search scope down will result in a much faster result.  To do this we can append a path to our previous command.  

\begin{Verbatim}[frame=single,fontsize=\relsize{-3}] 
john@akira:~/coderepo$ git log -SChange2 my_first_committed_file
john@akira:~/coderepo$ git log -SChange2 my_second_committed_file
commit 6ca160c7226731bf80973fc5bc81f6b9beda7795
Author: John Haskins <john.haskins@tamagoyakiinc.koala>
Date:   Mon Feb 21 20:59:32 2011 +0000

    Finished adding initial files
john@akira:~/coderepo$ 
\end{Verbatim}

If you remember from our committing back in Week 2, we added the string \texttt{Change2} to the second file but not the first.  So the first time we run this command, it fails, as we are searching against \texttt{my\_first\_committed\_file}.  The second time we run it, we are searching against \texttt{my\_second\_committed\_file} and this is where we see a result.  Commit 6ca16 contains the commit we are looking for.  

\begin{framed}
\subsubsection{Changing your identity}
Particularly when working with other people, or when publishing your repository to a public location, it's a good idea to make sure people know who you are and how to get in contact with you.  Every time you make a commit to a repository, Git gives the opportunity to take note of who posted the commit.  When you first install Git, it probably won't have the correct information in there for you, so it's important that you take the time to set this up.

To set up your name and email address, we need to modify the \texttt{gitconfig} again.  

\begin{Verbatim}[frame=single,fontsize=\relsize{-3}] 
$ git config --global user.name "John Haskins"
$ git config --global user.email 
"john.haskins@tamagoyakiinc.koala"
\end{Verbatim}

That's it.  Now by default, Git will use this setting whenever you commit to a repository, unless you override it by locally modifying the repository's \texttt{.gitconfig}.
\end{framed}

\clearpage

\section{Summary - John's Notes}
\subsection{Commands}
\texttt{git log} - Return a navigatable list of commits to a repository

\texttt{git log -S<string>} - Show all commits that either introduced or removed a particular string from the repository

\subsection{Terminology}
\textbf{Branch} - A way of working on the same set of code in parallel without modifications overlapping

%git log -p
%DIFFs
%Patching

%git show master:<path>
%git diff -- stylesheet.css master:stylesheet.css
%git checkout -b <branch>
