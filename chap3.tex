% chap3.tex - Week 3
\cleardoublepage
%\phantomsection
\chapter{Week 3}
\section{Day 1 - ``But how do I see what's going on?''}
\subsection{Logging in Git}

Perhaps the best feature of a version control system is the level of accountability that it offers if set up correctly.  What do we mean by this?  People often mistake the word \textbf{accountability} for the word \textbf{blame}.  This is not true at all.  Accountability is key in understanding the events that led up to a particular bug being introduced, or a situation occuring.  How this is dealt with, is a up to the management teams, but accountability should not be something that is revered, it should be something that is looked upon as a tool to help define the cause of a problem.

By its very nature, a version control system is also a logging system.  Every time we committed something into the repository in the last chapter, we supplied a log message.  In fact, if we don't supply a log message, let us see what happens.

\begin{Verbatim}[frame=single,fontsize=\relsize{-3}] 
pete@satsuki:~/coderepo$ git status
# On branch master
# Untracked files:
#   (use "git add <file>..." to include in what will be 
committed)
#
#	my_third_committed_file
nothing added to commit but untracked files present (use "git 
add" to track)
pete@satsuki:~/coderepo$ git add my_third_committed_file
pete@satsuki:~/coderepo$ git commit -a -m ''
Aborting commit due to empty commit message.
pete@satsuki:~/coderepo$ 
\end{Verbatim}

So, Git will actually not allow you to commit with a blank message.  This is actually fantastic news, as people are far less likely to write a useless message than they are a blank one.  It is very important that when using a version control system, that you write in a useful commit message.  If you fixed a bug, say so.  If you added a new function, why not put that in too.  When someone wants to find out what a certain commit was for, or even when you come back to the project six months later and realise you've forgotten everything, log messages are crucial in piecing back together a history of development.

\begin{trenches}
``I know John, and next time I will make a note of it, but right now, I'd really like to know where this file got changed, specifically when this function was introduced.''  Klaus pointed at the piece of paper containing a print out
\end{trenches}
