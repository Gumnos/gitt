% chap7.tex - Week 7
\cleardoublepage
%\phantomsection
\chapter{Week 7}

\section{Day 1 - ``Now let's work together''}
\subsection{Pure collaboration}

How to collab, pushing, rebasing, merging, patches

\section{Day 1 - ``Networking with a difference''}
\subsection{Pushing across a LAN}

Now we have a complete copy of our repository in another location.  At the moment we have created this clone on the same machine that our original is.  This isn't really a very good idea for backup purposes.  Git supplies several means with which to talk to a remote machine, but by far the most common of these is to utilise the SSH protocol.  SSH is a secure, encrypted way to communicate with a remote repository.  

If we assume that for a moment that our user john has now moved to another machine

In the next step, we are assuming that we have another machine where our user john has SSH access, and we are going to clone our repository there.
