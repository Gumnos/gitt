% chap5.tex - Week 5
\cleardoublepage
%\phantomsection
\chapter{Week 5}

\section{Day 1 - ``Conflicting information''}
\subsection{What to do when it all goes wrong}

The team have been playing with branches for a few days now and are beginning to settle into the idea of branching often.  As you can see, in Git, a branch is a really simple device for allowing experimentation and development.  As the implementation of branches is so simple, it is also really fast to switch between branches.  If you have been keeping up with the \emph{After Hours} sections, you can hopefully see that when switching branches, Git only needs to alter what is different between one working copy and another.

Let us see what new issues the team runs into during Week 5.

\begin{trenches}
``Hmm,'' the puzzled noise reverberated round Klaus' corner of the office.  Mumbling commenced.  ``So if I did that, then why is that not \ldots I mean I didn't think \ldots that's probably the problem Klaus \ldots sill 'luser'.''  He chuckled to himself, hardly noticing the looming figure of John.

``You got a conflict,'' said John matter of factly.

``I can see that,'' came Klaus' reply.  ``The problem is how do I solve it''

``What were you doing?''

``Well I had a few branches I had been working on and I was trying to merge them in.''

``Figures,'' said John.  
\end{trenches}

Conflicts are bound to happen at some point.  Even the best project processes in the world will sometimes end up with a conflict that has to be resolved.  We should, at this point, clarify what a conflict actually is.  A conflict occurs when an attempt is made to reconcile multiple changes to the same section of a file at the same time.  This usually means that the same line of code is modified by two different parties at the same time.  Now that we have several branches in our test repository, let us merge them back into master and see if we come up with a conflict.



%THIS IS TOO CONFUSING - NEED TO SIMPLIFY

\begin{Verbatim}[frame=leftline,framerule=1mm,fontsize=\relsize{-3}] 
john@akira:~/coderepo2$ git status
# On branch master
# Untracked files:
#   (use "git add <file>..." to include in what will be committed)
#
#	temp_file
nothing added to commit but untracked files present (use "git add" to track)
john@akira:~/coderepo2$ git merge wonderful 
Merge made by recursive.
 newfile1 |    1 +
 newfile2 |    1 +
 2 files changed, 2 insertions(+), 0 deletions(-)
john@akira:~/coderepo2$ git merge zaney 
Auto-merging newfile1
CONFLICT (content): Merge conflict in newfile1
Auto-merging newfile2
CONFLICT (content): Merge conflict in newfile2
Automatic merge failed; fix conflicts and then commit the result.
john@akira:~/coderepo2$ cat newfile1
A new file
<<<<<<< HEAD
and some more changes
=======
and some awesome changes
>>>>>>> zaney
john@akira:~/coderepo2$ cat newfile2
Another new file
<<<<<<< HEAD
and a new feature
=======
and some more awesome changes
>>>>>>> zaney
john@akira:~/coderepo2$ ls
another_file  my_third_committed_file  newfile1  newfile2  temp_file
john@akira:~/coderepo2$ nano newfile1

Error reading /home/john/.nano_history: Permission denied

Press Enter to continue starting nano.

john@akira:~/coderepo2$ cat newfile1 
A new file
and some more awesome changes
john@akira:~/coderepo2$ nano newfile2

Error reading /home/john/.nano_history: Permission denied

Press Enter to continue starting nano.

john@akira:~/coderepo2$ git status
# On branch master
# Unmerged paths:
#   (use "git add/rm <file>..." as appropriate to mark resolution)
#
#	both modified:      newfile1
#	both modified:      newfile2
#
# Untracked files:
#   (use "git add <file>..." to include in what will be committed)
#
#	temp_file
no changes added to commit (use "git add" and/or "git commit -a")
john@akira:~/coderepo2$ git add newfile1
john@akira:~/coderepo2$ git add newfile2
john@akira:~/coderepo2$ git status
# On branch master
# Changes to be committed:
#
#	modified:   newfile1
#	modified:   newfile2
#
# Untracked files:
#   (use "git add <file>..." to include in what will be committed)
#
#	temp_file
john@akira:~/coderepo2$ git commit -m 'Fixed conflicts in zaney merge'
[master a553c44] Fixed conflicts in zaney merge
john@akira:~/coderepo2$ !git log
git commit -m 'Fixed conflicts in zaney merge' log
error: pathspec 'log' did not match any file(s) known to git.
john@akira:~/coderepo2$ git log --graph --pretty=oneline --all --abbrev-commit --decorate
*   a553c44 (HEAD, master) Fixed conflicts in zaney merge
|\  
| * 4fdfe36 (zaney) Made another awesome change
| * 46c2390 Made an awesome change
* |   1db26b3 Merge branch 'wonderful'
|\ \  
| * | f0cbf32 (wonderful) Fantastic new feature
| |/  
* | 725c961 Continued Development
* | 6ab599d Important Update
|/  
* f91f31e Added another file
* 8d5140a (v2.0) Added two new files
* b5c84a5 Removed a few files
* fa65f06 (v1.0a) Messed with a few files
* 6ca160c Finished adding initial files
* e86ddea (v0.9) Made a few changes to first and second files
* 8820692 My First Ever Commit
john@akira:~/coderepo2$ gitk --all
^C
john@akira:~/coderepo2$ 
\end{Verbatim}

\section{Day 1 - ``No backup?''}
\subsection{Attack of the clones}

We now know about basic branching and merging.  At this stage, there is on important topic we must cover briefly, and this is subject of cloning.  Cloning allows you to make a complete copy of your repository, including all of its history and all of the branches.  Let us take a look at a situation which could make use of cloning.

\begin{trenches}
``So John,'' started Rob, ``Just how are we backing up the repository at the moment?''

John thought for a moment before replying.  He knew what Rob was getting at, but he hadn't expected Rob to bring the question up in front of Markus.  In truth he had forgotten all about it.  He turned to Markus.

``I'll be honest Markus.  Currently the repository isn't being backed up, but then we are running in parallel with the old system, so it would be the end of the world if we lost it.''

Markus nodded and smiled.  It seemed that John had gotten away with it for now and with that, Klaus shot Rob a piercing glance.  

``Team,'' began Markus, ``we need a definitive way of backing up the new Git repository, and I'd like it done before the end of the day.''  He pointed at a document on the table.  ``This project document has been approved by Wayne, and in there it states we will have a defined backup strategy.  Please don't let me down.''

\begin{center} * * * \end{center}

``Rob you really got to be careful about things like that,'' Klaus said to one of the younger members of the team.  ``You really showed John up in there.''  

``Yeh,'' said Rob, ``I realise that now.''  He stood with his back against the wall and tapped his fingers against the painted surface.  ``I was wondering, do you think John would let me look at the backup system for him as a way of apologising.''

Klaus smiled, it seemed Rob was finally understanding things, ``Go ask him,'' said Klaus, ``He's so snowed under with the BurnForce release that he'll probably let you implement it too''

``Right'' nodded Rob and off he went.
\end{trenches}

So cloning is an excellent way of taking a copy of our repository, in essence it is a simple way of taking a backup of our repository and keeping it up to date.  Obviously we could just take the files and copy them, but a better way of doing this is by utilising the \texttt{git clone} command.  When cloning a repository, we take then entire structure and replicate it, creating an exact copy of the data in an alternative location.  Well, that's what cloning means isn't it?

The git clone tool doesn't just copy the data though, it does several other things.  Let us create a clone of our test repository to another local location.  In this case, we are going to clone the repository into a folder called \texttt{coderepo-clone}.

\begin{Verbatim}[frame=leftline,framerule=1mm,fontsize=\relsize{-3}] 
john@akira:~$ git clone coderepo coderepo-clone
Initialized empty Git repository in /home/john/coderepo-clone/.git/
john@akira:~$ 
\end{Verbatim}

\begin{Verbatim}[frame=leftline,framerule=1mm,fontsize=\relsize{-3}] 
john@akira:~$ cd coderepo-clone
john@akira:~/coderepo-clone$ ls
another_file  my_third_committed_file  newfile1  newfile2
john@akira:~/coderepo-clone$
\end{Verbatim}

\begin{Verbatim}[frame=leftline,framerule=1mm,fontsize=\relsize{-3}] 
john@akira:~/coderepo-clone$ git status
# On branch master
nothing to commit (working directory clean)
john@akira:~/coderepo-clone$ 
\end{Verbatim}

Let us take a few moments to see what has happened here.  It would appear that our data has been copied successfully.  If you were to run a \texttt{git log} on this dir, you would see all the previous log messages for all our previous commits.  You can also see that if we run \texttt{git status} that we are on the same branch here that we were in our repository when we left it.  In this case, we are in the \textbf{master} branch.

\begin{trenches}
``Klaus, this doesn't make sense though.  I cloned my repo, but my branches have vanished,'' started Rob, rather worriedly.

``I'm sure they haven't gone anywhere,'' shouted John.  

Rob's reply was quick and certain, ``They have!  Run a git branch on your clone and see for yourself''

``Oh!''

Rob smiled the smug smile of a man who is pleased he was right.

\begin{center} * * * \end{center}

``Rob?'' asked John tentatively, aware of his previous state of anguish.

His reply indicated a less than stressed demeanour, ``Sup, dude?''.

John was pleasantly suprised, ``Did you sort out the branch issue?''

Rob begain walking over to John, ``Yeh, try running git branch with the r parameter.''

\end{trenches}

You may be thinking that we now have a detached copy of everything in our repository.  That's not exactly accurate.  Let us run the command that Rob suggested and see what is happening.  

\begin{Verbatim}[frame=leftline,framerule=1mm,fontsize=\relsize{-3}] 
john@akira:~/coderepo-clone$ git branch
* master
john@akira:~/coderepo-clone$ git branch -r
  origin/HEAD -> origin/master
  origin/master
  origin/wacky
john@akira:~/coderepo-clone$ 
\end{Verbatim}

Whilst we have all the objects in our repository, we do not yet have our branches set up locally.  They are available, but they have not been setup.  To explain this we need to introduce the concept of remote tracking branches.  We are familiar with local branches.  Local branches allowed us to make commits locally in a safe and separated environment.  Remote tracking branches are links to remote branches which allow us to track the development of that branch and bring it in to a local one if we so desire.

With a remote tracking branch, we can pull in the changes from the remote repository

Now we have a complete copy of our repository in another location.  At the moment we have created this clone on the same machine that our original is.  This isn't really a very good idea for backup purposes.  Git supplies several means with which to talk to a remote machine, but by far the most common of these is to utilise the SSH protocol.  SSH is a secure, encrypted way to communicate with a remote repository.  

If we assume that for a moment that our user john has now moved to another machine

In the next step, we are assuming that we have another machine where our user john has SSH access, and we are going to clone our repository there.

%POSSIBLY Talk about the SHARED and REFERENCE in a breakout
%POSSIBLY TALK ABOUT SSH
%ssh://[user@]host.xz[:port]/path/to/repo.git/




%branches
%merge
%remote tracking
%Itt section about not having anything backed up

%Parallel devving - ie...use both git and old for a while

% git show v1.0.0^{tree}
