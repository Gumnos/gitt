% setup.tex - Setup
\cleardoublepage
\chapter{Setting up}
\section{Making sure you have everything}
\subsection{Obtaining Git}
\index{Git!obtaining}Of course the most important tool we are going to need in this journey is Git itself and obtaining Git depends on the operating system you are using.
Though this book uses the Linux operating system throughout its examples,
all of the Git functionality described in this book can be performed no matter which operating system you choose.
You may find a few Linux commands used to perform operations like listing directory contents or echoing strings into files.
You operating system will likely have some functions that are similar, but these are not recorded here in the book.

\subsubsection{Windows}
\index{Git!on Windows}\index{msysgit}For the Windows operating system, the easiest way to obtain Git is to use the \emph{msysgit} package.
This package contains several helpful options other than the main Git binaries.
\emph{msysgit} includes two context menus called \emph{Git GUI Here} and \emph{Git Bash Here}.
These components attach themselves to the right-click menu in Windows so that if you right click on a folder that contains a Git repository,
you can work on it either in a command line, or graphical way.
This will be discussed in more detail later in the book.
During the installation of the \emph{msysgit} package, users are generally asked to make two choices, one is regarding their PATH setup and the other is to do with line endings.
For both of these it is recommended, at least at this stage, two choose the default options.
The Git implementation on Windows is currently considerably slower than implementations on Unix based systems.
However, it is still operates very well on the Windows platform.
The \emph{msysgit} package is available from http://gitscm.org.

\subsubsection{Linux}
\index{Git!on Linux}\index{native}For the Linux operating system, most Linux distributions come with Git packaged in some way.
In Ubuntu for example, one can install Git simply by running the command \texttt{sudo apt-get install git}.
There are many extra items on Linux that you can install that are related to Git, however for the purpose of this book, just the core package and the gui are all that are required.
If you cannot find a package for your distribution, you can always either compile it from source, or take a look at http://gitscm.org to see if there is one available for your system.

\subsubsection{MacOS}
\index{Git!on MacOS}For MacOS
If you are using the MacOS platform, Git can be found as a download from http://gitscm.org.
