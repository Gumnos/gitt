% intro.tex - Introduction
\cleardoublepage
%\phantomsection
\addcontentsline{toc}{chapter}{Introduction}
\addcontentsline{toc}{section}{How this book works}
\chapter*{Introduction}
\section*{How this book works}
Welcome to Git In The Trenches or GITT, a book designed to help you both apply and understand the subtleties of Git, perhaps the most powerful version control system in use today.  This book isn't supposed to be purely a technical reference, moreover it is hoped that the experiences and scenarios that you encounter will help give ways to apply Git in practical applications.  Git is a hugely powerful system and once harnessed you are most likely going to wonder how you managed without it.

GITT follows the lives of some developers at a fictional company called Tamagoyaki Inc.  They are a small software outfit who write bespoke software for people.  It may be that you work for a company that is very similar to Tamagoyaki Inc and you are looking to implement a version control system for your own company, or it could be that you have been using a version control for a long time.  Regardless of which box you fit into, GITT should provide you with some useful knowledge in a way that is designed to help you remember the scenarios and their associated solutions.

The book will follow the lead developer John, as he struggles to bring the company into line by implementing a version control system.  It's not something he's ever really used in earnest and he feels a little out of his depth.  It is hoped that your confidence and knowledge about both version control systems, and Git in particular, will grow whilst reading GITT.  

The chapters are presented as weeks during the implementation of Tamagoyaki Inc's VCS.  Each chapter spells a new week in the project and you will follow the life of John and his colleagues as they solve problems and learn tricks of the Git trade.  As well as presenting and solving common issues, the book will also be littered with breakout boxes, intended to tell you exactly what is happening inside Git at each stage.  This is intended to further your knowledge and understanding of this powerful piece of software.  At the end of each chapter are "John's Notes" which should build into a quick reference guide, also included with the book.

Often knowing the commands is not the only piece of the puzzle.  A good understanding of the underlying system, how it reacts when you press that all important <<Enter>> button, is essential if you want to be able to hold your cool in a crisis.  So let's start at Week 1 and find out why Tamagoyaki Inc even need a version control system.
